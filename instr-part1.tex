\section*{Project I: Topic ID and Overview (due \textcolor{red}{Sept 19})}

%This is just a placeholder for now

The first step in your semester-long research proposal development is to select a topic area in the mechanics of soft materials that's sufficiently interesting to you. 
It may be helpful to think of the proposal-writing process as the following. 
You want to study something that you are especially interested in, but you don't yet have the resources that you need to pursue this fully. 
Your job is (eventually) to communicate what it is you want to study, why it's worthwhile to be studied, and enumerate all of the reasons it's in some benefactor's interest to provide you the support that you need.
The particular benefactor we will leverage is the National Science Foundation, which cares about making fundamental ``vertical'' advances in fields (known as their \textit{Intellectual Merit} criterion) and having their funded projects improve society (known as the \textit{Broader Impact} to society criterion). 

Aim for approximately 500 words of total text, such as to reflect the important three Cs: \textbf{\textit{clear}}, \textbf{\textit{concise}}, and \textbf{\textit{compelling}}.
Your submission should be structured in three sections as separated below, and should address the following points: 

\begin{enumerate}
\item \textbf{Statement of Research Interest (why you personally want to study the subject)}
\begin{itemize}
\item Describe an area or phenomenon in the mechanics of soft materials that you find compelling. 
\item What motivates your interest and pursuit of this subject (e.g., your current or developing expertise, research interests, or otherwise)? \textit{Note: This may be more personal or anecdotal and is for my own understanding of your topic selection!}
\end{itemize}
\item \textbf{Intellectual Merit (why it is objectively worth delving deeper)}
\begin{itemize}
\item Describe, to someone with expertise in mechanics but perhaps not your system of interest, the core scientific principles underpinning (or perhaps, enabling development in) your topic of interest. 
\item Given the course syllabus, how will particular material we will cover this semester relate to what you propose? What background information do you need to do not just a good, but great, job in proposing something interesting? 
\end{itemize}
\item \textbf{Broader Impact (who, or what, does studying this area benefit?)}
\begin{itemize}
\item How might advances you envision in this area be impactful beyond your own interest? 
\item What does a ``winning scenario'' in this area look like? Briefly describe who might benefit (e.g., particular industries, health/science sectors, the public) and how that could plausibly happen.
\end{itemize}
\end{enumerate}

\emph{A strong submission will clearly illustrate your personal goals with this project, and show how your interest could manifest as advances in the broader field and society.}

\textbf{Statement of Research Interest} \\
Mechanical gradient soft materials (MGSMs) abound throughout the natural world. From the human eye to tendon-bone mineralization, natural tissues often use \textit{continuous} changes in modulus to utilize inherent benefits of more durable, hard materials and more compliant, soft materials without the stress-concentrations that would be induced if the material changes moduli in an abrupt manner. In the world of man-made materials and structures, process-driven mechanical gradients are inherent in many manufacturing processes. The focus of this proposal will be resin 3D-printing where gradients in curing dosage in each layer, thermal inhomogeneities, and many other process-driven gradients manipulate the mechanical properties and therefore reliability of the final part. Speaking from personal experience, these sources of error fascinated me an I first began printing self-healing photopolymers with Air Force Research Laboratory during my Junior year of undergrad. This interest was only amplified around 2015 as Carbon 3D and a few academics, including Jerry Qi, began publishing so-called "grayscale" prints that used light-flux gradients in their printers to achieve engineered modulus distributions throughout their 3D prints. Despite these major advancements in resin 3D printing tech, the solid mechanics techniques for \textif{verifying} the volumetric local mechanical properties of the print remain severely underdeveloped.

\textbf{Intellectual Merit} \\
A wealth of inverse methods exist for characterizing the mechanical properties of heterogeneous soft materials at finite strains. Leveraging rich full-field kinematic data from techniques such as digital image correlation, digital volume correlation, and magnetic resonance imaging these techniques generally use an energy balance via the virtual fields methods or an error of the measured and simulated displacement fields for a selected model with given parameters. Within the last year, several papers have begun using these techniques to characterize heterogeneous materials, however, the use of these techniques with hyperelastic soft materials remains in its infancy. I propose using a high-area rapid printing (HARP) design 3D-printer capable of printing grayscale hydrogels with a greater than 1 order of magnitude tunable modulus. Adapting the methods I learn in this course, I will characterize the hyperelastic properties of simple grayscale layer-less specimens (e.g., linear gradient). Using stress-relaxation and creep tests, viscous characterization of the grayscale material will attempt to detect differences beyond stiffness for the first time. 

\textbf{Broader Impact}
Beyond the simple academic pursuit of characterizing the time-dependence in grayscale polymers via a robust inverse methods, many industries could benefit from a careful understanding of the tunable viscous properties that can be achieved from grayscale printing. Given access to the novel method I am proposing, polymer chemists will be much more likely to develop novel resins with significantly tunable viscous properties. Advances from this work will boost safety of biological implants and hydrogel drug-delivery devices by ensuring rate-dependent loading scenarios are well understood. Increased soft component reliability could also boost profitability from increased life-span across industries from high-performance sports products to custom prosthetics. 