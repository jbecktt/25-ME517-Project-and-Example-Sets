\section*{Project II: Literature Review (due Oct 3)}

The second step in your semester-long research proposal development is to contextualize your problem within the current field of your choice, demonstrate your understanding of the state-of-the-art, and identify something we don't yet understand but need to---this is the gap your (hypothetical) proposed project would seek to fill.

You'll execute this portion of the project as an outline. 
This outline does not need to be long! 
It does, however, need to be very clear, as you'll be expanding on it later. 
The sections should be the following:

\renewcommand{\outlinei}{enumerate}
\renewcommand{\outlineii}{itemize}
\begin{outline}
    \1 \textbf{Introductory context}
        \2 Add one or two bullet points briefly framing the history of your topic and its significance. 
        \2 Citations here are important---there should be several well-placed citations in each of these bullet points, and good review papers are especially helpful to lean on for context. 
        \2 \textit{\textbf{Example}: While gradient materials occur widely in nature---the structurally protective gradient of squid beaks [1], junctions between ligaments and bones [2], and the byssus threads that hold mussels to rocks [3] to name a few---polymeric gradient materials were only first considered in an engineering context in 1972 [4].}
    \1 \textbf{The state of the field}
        \2 In one or two bullet points, explain how things are done in the area of your project at the present moment. 
        \2 This can include any of experimentation, computational methods, and/or theory.
        \2 \textit{\textbf{Example 1}: Compositional gradients in engineered materials are typically produced in one of three ways: spatially (1) varying polymer crosslink density, e.g. using ultraviolet light-sensitive reactive groups [5, 6], (2) seeding of micro- or nanoparticles using centrifugation [7], electric fields [8], or other methods [9], or (3) using porosity to create structural gradients using dissolvable template-making materials [10].}
        \2 \textit{\textbf{Example 2}: Additive manufacturing has emerged as perhaps the best option for making complex functionally gradient soft materials on the order of cm or larger [11-14].}
    \1 \textbf{The Big Gap}
        \2 In one or two sentences, what is it that we don't know, and why isn't it solved? 
        \2 \textit{\textbf{Example 1:} The principal limitation of the first three techniques is scalability to larger sizes.}
        \2 \textit{\textbf{Example 2:} Arguably, the most challenging barrier for widespread production of gradient materials is the combination of sample repeatability and a comprehensive lack of validation options.} 
\end{outline}

\textit{A strong literature review outline will succinctly illustrate the essential context of the problem, the current best knowledge in the area, and the critical gap in knowledge restricting further advances or implementation, and should cite approximately 10-15 references with a \LaTeX ~bibliography.}

\bigskip

For context, since last time I have (at Jon's request) opted to focus more on inverse solving for spatial distributions in viscous properties in tandem with the nonhomogeneous moduli from full-field data. This would clearly be challenging, however, it would be incredibly novel. In terms of my vision, if this was actually completed I would probably opt to conduct this study on a soft elastomeric material, but early studies would focus on small strains and eventually building on to finite strain viscoelastic characterization. 

\textbf{Introductory Context}
\begin{itemize}
    \item Functionally gradient soft materials (FGSM) abound in nature from fish-scales \cite{Yang2019} to gradient orientations of cellulose microfibrils in tree branches \cite{liu2017functional} oftentimes yielding structures that offer the durability and strength of more rigid materials with the impact absorption and high flexibility from soft materials. 
    \item Since the advent of additive manufacturing (AM) in the early 1980's \cite{Kodama1981} and subsequent work printing metals \cite{beaman1990selective}, the main use case has been manufacturing low-volume and/or complex form factor components \cite{Wong2012}. Primarily from the 2010's onward, several polymer AM techniques \cite{Pragya2023} were modified to manufacture detailed functionally gradient plastic and elastomer device providing an additional edge over traditional manufacturing techniques. 
\end{itemize}

\textbf{The State of the Field}
\begin{itemize}
    \item Primary methods for inverse solving material property distributions from full-field deformation measurements include physics-based methods including VFM \cite{grediac1989principe} and FEMU \cite{mottershead1993model} as well as data-driven approaches such as physics-informed neural networks \cite{GHAFFARIMOTLAGH2025117650}. Recently these techniques have been expanded to characterizing nonhomogenous materials at linear elastic \cite{Mei2021} and hyperelastic \cite{shojaee5294317automated} regimes. 
    \item In the world of grayscale 3D printing \cite{Kuang2019,Peterson2016} and even more broadly gradient multi-material forms of ink jetting \cite{Müller2014} and direct ink writing \cite{Young2024} have emphasized tunability in solely Young's modulus. Broadly speaking, most prints are validated quite crudely by quantifying their time-dependent macroscopic response to stimuli in 4D prints \cite{ANDREU2021102024} and simple tests of relative stiffness in FGSM prints such as deforming parts of the device by hand \cite{Yue2023}. 
\end{itemize}

\textbf{The Big Gap}
\begin{itemize}
    \item Despite recent advances in inverse methods for nonhomogenous specimens, the characterization of spatially-varying viscoelastic response largely remains absent in the literature. Several factors likely contribute to the lack of work in this area including the relatively recent dawn of methods characterizing FGSMs at quasi-static rate and the lack of full-field measurements studies conducted on novel grayscale prints stemming from the lack of sufficient collaboration between 3D printing and solid mechanics communities. 
\end{itemize}

\bibliographystyle{unsrt}
\bibliography{references}